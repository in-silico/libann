\documentclass{article}

\usepackage{mathtools}
\usepackage{graphicx}
\usepackage{subfig}
\usepackage{verbatim}
\usepackage{algpseudocode}
\usepackage{natbib}
\usepackage{url}

\begin{document}

\title{Paper currency recognition with color histograms}
\author{Sebastian Gomez \\ Tatiana Lopez}
\maketitle

\begin{abstract}
This document shows an approach to make a color histogram based classification of paper currency from an image.
%todo:Terminar abstract
\end{abstract}

\section{Introduction}
For blind people in many countries it is hard to recognize the denomination of their local paper currency because
there are not enough non optic features in the bills. It would be useful for them to have automated software that
can recognize the currency denomination from an image.

There are several publications on these kinds of systems. %todo: Mention them

Our approach uses only color information, but the accuracy might be improved by adding features that make use of
texture information.

\section{Methodology}

The classification system consist of a feature extraction part, whose responsibility is to compute a vector of
fixed dimensionality. That feature vector is then feeded to a machine learning algorithm to classify the input
into one of the corresponding classes.

%feature extraction
RGB is a color model that expresses the color of each pixel as a vector in a 3-Dimensional space, whose components
represent the proportion of red, green and blue respectively. This model is widely used for screens, cameras and
other optical devices but its disadvantage is that it mixes color and brightness.

There are other color models such as Yxy from cielab that separate brightness and color in different channels. In
this system we used Yxy dropping the brightness channel. This decision was taken to generate brightness invariance
in the feature vectors. %continue the idea

%classification systems

\section{Results}

\section{Conclusions}

\end{document}
